%!TEX root = ../Thesis.tex


\chapter{Python Scripts}

Appendix A includes all the developed Python scripts throughout this study. 


%C.1includes the scripts for the simulation and damage assessment of the 3D steel frame,
%while C.2 includes the corresponding scripts for the offshore wind turbine.

\newpage
\begin{lstlisting}[language=Python, caption = Python Script that create plots for all different scenario's, label=lst:plots]
#Python Script that create plots for all different scenario's
#Import all the important packets
import pandas as pd 
import matplotlib.pyplot as plt 
import seaborn as sns
import matplotlib as mpl
import numpy as np
import matplotlib.cm

#Read all the data from the file
df1=pd.read_excel('DataScenarios.xlsx', index_col=0)

df_list=[]
for i in df1['dfid'].unique():
    x='df'+str(i)
    x=df1[df1['dfid'] == i]
    df_list.append(x)
    
sns.set(style="darkgrid")

#add colors
colors = ['#4daf4a','#e41a1d']
mylabels = ["Successful","Unsuccessful"]
       
fig, ax = plt.subplots()
###########################################################################
for df in df_list:

    for percentage in df.Percentage:
        plt.pie([percentage,100.0 - percentage], autopct='%.2f%%', shadow=True, startangle=90, colors = colors, labels = mylabels)
        plt.legend()
        plt.xlabel('Successful Transmissions')
        plt.show() 
    index = range(1,len(df)+1)
    array = []
    for i in index:
        array.append(i) 

    fig, ax = plt.subplots()
    pps = plt.bar(array, df.Percentage, align = 'center', alpha = 0.5, color = 'maroon', width = 0.5)
    plt.xticks(array)
    plt.ylabel('Successful Transmissions')
    plt.title("Successful Rate in Different Experiments")
    for p in pps:
        height = p.get_height()
        ax.text(x=p.get_x() + p.get_width() / 2, y=height+.10, s="{}%".format(height), ha='center',  weight='bold')
    plt.show()
    
\end{lstlisting}
\newpage
\begin{lstlisting}[language=Python, caption = Python Script that search for duplications\, check for successful transmissions and find out the overall and individual node PDRs, label=lst:DER]
#Python Script that search for duplications, check for successful transmissions and find out the overall and individual node DERs.
#Import all the important packets
import re
import subprocess
import pandas as pd
from collections import Counter

class bcolors:
    HEADER = '\033[95m'
    OKBLUE = '\033[94m'
    OKCYAN = '\033[96m'
    OKGREEN = '\033[92m'
    WARNING = '\033[93m'
    FAIL = '\033[91m'
    ENDC = '\033[0m'
    BOLD = '\033[1m'
    UNDERLINE = '\033[4m'

df = pd.DataFrame(columns=['intersections', 'total', 'Percentage'])
NODE_NUM = 5

# for all nodes
for node in range(NODE_NUM): 
    # get output from grep commands and transform into utf8
    data = subprocess.Popen(['grep "LoRaGWRadio Reception ended: successfully for" logn5tractormobgw1-29.elog | grep "transmissionId"'], shell=True, stdout = subprocess.PIPE)
    data1 = subprocess.Popen([f'grep "Transmission started:" logn5tractormobgw1-29.elog | grep "transmitterId = {node}" | grep "id"'], shell=True, stdout = subprocess.PIPE)
    output = data.stdout.read().decode("utf-8")
    output1 = data1.stdout.read().decode("utf-8")

    rids = [] # receiver ids
    intersection = [] # the match between rids and tids
    tids = set() # node duplicates distinct and unordered / transmission ids

    print('{}Searching logfile...{}'
                .format(bcolors.HEADER, bcolors.ENDC))
    # split grep command output on transmissionid and get second part
    for i, item in enumerate(output.split("transmissionId")):
        if i == 0:
            continue
        
        # get 1st and 2nd numbers with regular expressions
        tid = re.search(r'\d+', item).group()
        rid = re.search(r'\d+', item[10:]).group()
        
        # check for successful transmissions by matching the previous in the other grep command 
        for j, item in enumerate(output1.split("id")):
                if j == 0:
                    continue

                ida = re.search(r'\d+', item).group()

                if ida == tid :
                    #print(f'Node {node} send and Found ID: {ida}')
                    intersection.append(ida)


        #print(f'TID: {tid} \t RID: {rid}')
        rids.append(rid)
        tids.add(tid)


    # casting to a set to get unique values    
    intersection_set = set(intersection)
    print(f'Intersection on node {node}: {len(intersection_set)}')
    # Filing the DataFrame with per node PRR 
    df.loc[node] = [len(intersection_set), j, len(intersection_set) / j*100] # j is the total for each node 

# Export the dataframe to an excel 
df.index.name = 'node'
df.to_excel("output_nodes_stats.xlsx")

print(df)
print(f'\nCummulative successfull transmission count: {len(rids)}\nIndividual count: {Counter(rids)}')
print(f'Successfull non-duplicated transmissions: {len(tids)}')    

# Getting the total from mystats ( mystats.txt is created with a custom script embedded in the simulator code)
cmdout = subprocess.Popen(['cat mystats.txt'], shell=True, stdout = subprocess.PIPE)
cmdout_proc = cmdout.stdout.read().decode("utf-8")
total = int(re.search(r'\d+', cmdout_proc).group())

print(f'Total transmissions: {total}')

percentage = len(tids) / total * 100
print('{}{}Successfull transmission percentage: {:.2f}%{}{}'
            .format(bcolors.BOLD, bcolors.OKGREEN, percentage, bcolors.ENDC, bcolors.ENDC))
\end{lstlisting}




\newpage
\begin{lstlisting}[language=Python, caption = Python Script that visualize the results with standard deviation error bars, label=lst:ploteachnode]
#Python Script that visualize the results with standard deviation error bars.
#Import all the important packets
import pandas as pd 
import matplotlib.pyplot as plt 
import seaborn as sns
import matplotlib as mpl
import numpy as np
import matplotlib.cm


 
node_stats = pd.read_excel('tractor_merged.xlsx', index_col=[0])
# for x nodes
x=16
node_list=[]
for i in range(x):
  node_list.append(node_stats.loc[node_stats['node'] == i])

node_percentages=[]
for item in node_list:
  node_percentages.append(item['Percentage'].to_numpy())

node_averages=[]
for item in node_percentages:
  node_averages.append(np.mean(item))

node_stds = []
for item in node_percentages:
  node_stds.append(np.std(item))

nodes=[]
for i in range(x):
  nodes.append(str(i))

x_pos = np.arange(len(nodes))
CTEs = node_averages
error = node_stds

sns.set()
fig, ax = plt.subplots(figsize=(12, 6))
plt.ylim([0, 100])


df = pd.DataFrame(node_averages,columns=['Percentage'])
#print(df)
df.to_excel("final_percentages.xlsx")                    

pps = ax.bar(x_pos, CTEs, yerr=error, align='center', alpha=0.5, ecolor='maroon', capsize=10, width = 0.5)
for p in pps:
    height = p.get_height()
    ax.text(x=p.get_x() + p.get_width() / 2, y=height+ 4, s="{:.2f}%".format(height), ha='center',  weight='bold')

#ax.text(4.5, 100, 'Jain\'s fairness \nindex = 0.8744', size=15, color='purple')    
ax.set_ylabel('Packet Delivery Rate / PDR')
ax.set_xticks(x_pos)
ax.set_xticklabels(nodes)
ax.set_xlabel('Nodes')
ax.set_title('Successful Rate on each node')


# Save the figure and show
plt.tight_layout()
plt.savefig('bar_plot_with_error_bars.png')
plt.show()
 
\end{lstlisting}

\begin{lstlisting}[language=Python, caption = Python Script that calculates Jain's fairness index, label=lst:jindex]
#Python Script that calculates Jain's fairness index.
#Import all the important packets
import pandas as pd 
import matplotlib.pyplot as plt 
import seaborn as sns
import matplotlib as mpl
import numpy as np
import matplotlib.cm

#throughputs = [0.4244, 0.4942, 0.3314, 0.3140, 0.4128]
df1=pd.read_excel('output_nodes_stats_1gw.xlsx', index_col=0)
throughputs=df1['Percentage'].div(100).round(4).tolist()
throughputs

def jains_fairness_index(throughputs):
        
    n = len(throughputs)
    temp = sum([ (x**2) for x in throughputs])
    jains_index = sum(throughputs) ** 2 / (n * temp)
 
    return jains_index


jains_index = jains_fairness_index(throughputs)
print(jains_index)
\end{lstlisting}


All the files with the source code and the python scripts can be found in this \href{https://github.com/memorial4/LoRa_Mobility}{link}.


