%!TEX root = ../Thesis.tex
\chapter{Introduction}

Reasoning is the cognitive process of using existing knowledge to make inferences, create explanations, and assess statements rationally through logic \cite{InternetClassicsArchive}. It is fundamental to critical thinking, argumentation, and decision-making in various domains, from philosophy and science to public discourse and artificial intelligence (AI). However, human reasoning is prone to systematic errors, leading to logical fallacies—arguments that appear persuasive but are logically unsound.

Logical fallacies are prevalent in everyday life, appearing in casual conversations, advertisements, political debates, and social media. Some are harmless, such as, (source of the examples: Jin et al. \cite{jinLogicalFallacyDetection2022}):

\begin{itemize}
    \item \textbf{Faulty Generalization:} ``All tall people like cheese."
    \item \textbf{Circular Reasoning:} ``She is the best because she is better than anyone else."
\end{itemize}

However, fallacies are also intentionally exploited to manipulate opinions and spread misinformation, as seen in:

\begin{itemize}
    \item \textbf{Faulty Generalization:} ``Today is so cold, so I don’t believe in global warming."
    \item \textbf{Circular Reasoning:} ``Global warming doesn’t exist because the Earth is not getting warmer."
\end{itemize}

In an era of digital misinformation, detecting and mitigating logical fallacies is essential for rational discourse and evidence-based decision-making. Fallacy detection, once studied in philosophy and rhetoric, has now become a key challenge in computational linguistics, AI, and misinformation analysis.

Here I'll continue with the limitations and my work.




\noindent 
